

\documentclass[11pt]{article}
\usepackage{graphicx}
\usepackage{amsmath}
\usepackage{float}

\begin{document}

\title{Tarea 4: Metodos Comutacionales \\ 2018-20}% Force line breaks with \\

\author{Felipe Giraldo Grueso \\ 201631172}

\date{\today}

\maketitle

\section{Ecuaciones diferenciales ordinarias}
	\subsection{45 grados}
		\begin{figure}[H]
    			\begin{center}
    				\includegraphics[scale = 0.7]{ODE_45.pdf}
    				\caption{}
			\end{center}
    			\label{fig:45}
		\end{figure}

	\subsection{Todos los grados posibles}
		\begin{figure}[H]
    			\centering
    				\includegraphics[scale = 0.7]{ODE_todos.pdf}
    				\caption{}
    			\label{fig:todos}
		\end{figure}


\section{Ecuaciones diferenciales parciales}
	\subsection{Fronteras fijas}
		\begin{figure}[H]
    			\centering
    				\includegraphics[scale = 0.5]{PDE_fijas.pdf}
    				\caption{}
    			\label{fig:fijas}
		\end{figure}

	\subsection{Fronteras abiertas}
		\begin{figure}[H]
    			\centering
    				\includegraphics[scale = 0.5]{PDE_abiertas.pdf}
    				\caption{}
    			\label{fig:abiertas}
		\end{figure}

	\subsection{Fronteras periodicas}
		\begin{figure}[H]
    			\centering
    				\includegraphics[scale = 0.5]{PDE_periodicas.pdf}
    				\caption{}
    			\label{fig:periodicas}
		\end{figure}
	
	\subsection{Promedios}
		\begin{figure}[H]
    			\centering
    				\includegraphics[scale = 0.7]{PDE_promedio.pdf}
    				\caption{}
    			\label{fig:promedios}
		\end{figure}
	
\end{document}

